%%%%%%%%%%%%%%%%%%%%%%%%%%%%%%%%%%%%%%%%%
% Beamer Presentation
% LaTeX Template
% Version 1.0 (10/11/12)
%
% This template has been downloaded from:
% http://www.LaTeXTemplates.com
%
% License:
% CC BY-NC-SA 3.0 (http://creativecommons.org/licenses/by-nc-sa/3.0/)
%
%%%%%%%%%%%%%%%%%%%%%%%%%%%%%%%%%%%%%%%%%

%----------------------------------------------------------------------------------------
%	PACKAGES AND THEMES
%----------------------------------------------------------------------------------------

\documentclass{beamer}

\mode<presentation> {

% The Beamer class comes with a number of default slide themes
% which change the colors and layouts of slides. Below this is a list
% of all the themes, uncomment each in turn to see what they look like.

%\usetheme{default}
%\usetheme{AnnArbor}
%\usetheme{Antibes}
%\usetheme{Bergen}
%usetheme{Berkeley}
%\usetheme{Berlin}
%\usetheme{Boadilla}
%\usetheme{CambridgeUS}
%\usetheme{Copenhagen}
%\usetheme{Darmstadt}
%\usetheme{Dresden}
%\usetheme{Frankfurt}
%\usetheme{Goettingen}
%\usetheme{Hannover}
%\usetheme{Ilmenau}
%\usetheme{JuanLesPins}
%\usetheme{Luebeck}
%\usetheme{Madrid}
%\usetheme{Malmoe}
%\usetheme{Marburg}
\usetheme{Montpellier}
%\usetheme{PaloAlto}
%\usetheme{Pittsburgh}
%\usetheme{Rochester}
%\usetheme{Singapore}
%\usetheme{Szeged}
%\usetheme{Warsaw}

% As well as themes, the Beamer class has a number of color themes
% for any slide theme. Uncomment each of these in turn to see how it
% changes the colors of your current slide theme.

%\usecolortheme{albatross}
%\usecolortheme{beaver}
%\usecolortheme{beetle}
%\usecolortheme{crane}
%\usecolortheme{dolphin}
%\usecolortheme{dove}
%\usecolortheme{fly}
\usecolortheme{lily}
%\usecolortheme{orchid}
%\usecolortheme{rose}
%\usecolortheme{seagull}
%\usecolortheme{seahorse}
%\usecolortheme{whale}
%\usecolortheme{wolverine}

%\setbeamertemplate{footline} % To remove the footer line in all slides uncomment this line
%\setbeamertemplate{footline}[page number] % To replace the footer line in all slides with a simple slide count uncomment this line

%\setbeamertemplate{navigation symbols}{} % To remove the navigation symbols from the bottom of all slides uncomment this line
}

\usepackage{graphicx} % Allows including images
\usepackage{booktabs} % Allows the use of \toprule, \midrule and \bottomrule in tables

%----------------------------------------------------------------------------------------
%	TITLE PAGE
%----------------------------------------------------------------------------------------

\title[Euclidean Algorithm]{Exploring the Euclidean Algorithm} % The short title appears at the bottom of every slide, the full title is only on the title page

\author{Stephen Capps, Sarah Sahibzada, \& Taylor Wilson} % Your name
\institute[TAMU] % Your institution as it will appear on the bottom of every slide, may be shorthand to save space
{
Texas A\&{}M University \\ % Your institution for the title page
\medskip
\textit{Supurvisor: Sara Pollock} % Your email address
}
\date{\today} % Date, can be changed to a custom date

\begin{document}

\begin{frame}
\titlepage % Print the title page as the first slide
\end{frame}

\begin{frame}
\frametitle{Overview} % Table of contents slide, comment this block out to remove it
\tableofcontents % Throughout your presentation, if you choose to use \section{} and \subsection{} commands, these will automatically be printed on this slide as an overview of your presentation
\end{frame}

%----------------------------------------------------------------------------------------
%	PRESENTATION SLIDES
%----------------------------------------------------------------------------------------

%------------------------------------------------
\section{Introduction}
\subsection{Definition}
\begin{frame} $ $
\indent The Euclidean Algorithm is used to find the Greatest Common Divisor between any pair of whole numbers $\mathrm{p, q}$ such that $\mathrm{p>q}$.
\indent It follows that 
$$p = n_1*q + r_1$$ 
$$q = n_2*r_1 + r_2$$
$$.$$
$$.$$
$$.$$
$$r_{k-1} = n_{k+1}*r_k$$
Where \begin{center}$r_k = \mathrm{gcd(p,q)}$.\end{center}
\end{frame}
\subsection{Examples}
\begin{frame}
For example, here is the gcd$(42,36)$:
		$$\mathrm{gcd}(42,36) = 6:$$

\begin{center}--------------------------------------------------------------
\end{center}	
	\begin{equation}
		42 = 1 * 36 + 6
	\end{equation}
	\begin{equation}
		36 = 6 * 6 + 0
	\end{equation}
As you can see, it took $2$ iterations to complete the algorithm. This is what we will explore. Here are some more gcds and their iterations:

\begin{center}
\begin{tabular}{c|c}

$\mathrm{gcd}(\mathrm{p},\mathrm{q}) = \mathrm{d}$ & Iterations
\\
\hline
$\mathrm{gcd}(689,456) = 1$ & $6$\\

$\mathrm{gcd}(78,45) = 3$ & $5$\\

$\mathrm{gcd}(8394,238) = 2$ & $7$\\


\end{tabular}
\end{center}
\end{frame}

\section{Euclidean Algorithm Iterations and Results}

\subsection{What to explore}
\begin{frame}
Next, we decided to explore the distributions of these iterations:
Do most pairs take many iterations? What is the distribution?

The following graphs are the answer to these questions.


\end{frame}
\subsection{Distribution Results}
\begin{frame}
\begin{figure}
		
		\center \includegraphics[scale=.3]{2digit_iterationfreq.jpg}
		\center \tiny(Figure 1)
\end{figure}


\end{frame}

\begin{frame}
\begin{figure}
		
		\center \includegraphics[scale=.3]{4digit_iteration_freq.jpg}
		\center \tiny(Figure 2)
\end{figure}


\end{frame}

\begin{frame}
\begin{figure}
		
		\center \includegraphics[scale=.3]{10_digit_numbers.jpg}
		\center \tiny(Figure 3)
\end{figure}

\end{frame}

\begin{frame}
		\center \includegraphics[scale=.3]{100_digit_numbers_freq.jpg}
		\center \tiny(Figure 4)

\end{frame}

\begin{frame}
		\center \includegraphics[scale=.3]{1000_digit_numbers.jpg}
		\center \tiny(Figure 5)

\end{frame}

\subsection{Notes}
\begin{frame}
\begin{itemize}
\item in Figure 1, it must be noted that the lack of normality here is due to the lack of available bins. The size of this data set was no more than $4950$, and spanned across merely $9$ bins. As the number of bins needed increases, the more normal the graph becomes.
\item in Figure $5$, the inconsistencies in the normal distribution can be attributed to a too small a sample size. If given the computing power and time, one could compute all pairs less than $10^{1000}$. Note the increasing mean iterations as we climb the upper bound.
\end{itemize}

\end{frame}

\section{Introduction to Complexity Theory}

\section{Euclidean Algorithm Variations}

\section{Complexity Results}

\section{Introduction to Neural Networks}

\section{Attempts at using Neural Networks}

\section{Neural Networks Results}


\subsection{Subsection Example} % A subsection can be created just before a set of slides with a common theme to further break down your presentation into chunks



%------------------------------------------------



%------------------------------------------------



%------------------------------------------------



%------------------------------------------------

%------------------------------------------------

%------------------------------------------------



%------------------------------------------------

\begin{frame}[fragile] % Need to use the fragile option when verbatim is used in the slide
\frametitle{Verbatim}
\begin{example}[Theorem Slide Code]
\begin{verbatim}
\begin{frame}
\frametitle{Theorem}
\begin{theorem}[Mass--energy equivalence]
$E = mc^2$
\end{theorem}
\end{frame}\end{verbatim}
\end{example}
\end{frame}

%------------------------------------------------



%------------------------------------------------



%------------------------------------------------

\begin{frame}
\frametitle{References}
\footnotesize{
\begin{thebibliography}{99} % Beamer does not support BibTeX so references must be inserted manually as below
\bibitem[Smith, 2012]{p1} John Smith (2012)
\newblock Title of the publication
\newblock \emph{Journal Name} 12(3), 45 -- 678.
\end{thebibliography}
}
\end{frame}

%------------------------------------------------

\begin{frame}
\Huge{\centerline{The End}}
\end{frame}

%----------------------------------------------------------------------------------------

\end{document} 