%%%%%%%%%%%%%%%%%%%%%%%%%%%%%%%%%%%%%%%%%
% Beamer Presentation
% LaTeX Template
% Version 1.0 (10/11/12)
%
% This template has been downloaded from:
% http://www.LaTeXTemplates.com
%
% License:
% CC BY-NC-SA 3.0 (http://creativecommons.org/licenses/by-nc-sa/3.0/)
%
%%%%%%%%%%%%%%%%%%%%%%%%%%%%%%%%%%%%%%%%%

%----------------------------------------------------------------------------------------
%	PACKAGES AND THEMES
%----------------------------------------------------------------------------------------

\documentclass{beamer}

\mode<presentation> {

% The Beamer class comes with a number of default slide themes
% which change the colors and layouts of slides. Below this is a list
% of all the themes, uncomment each in turn to see what they look like.

%\usetheme{default}
%\usetheme{AnnArbor}
%\usetheme{Antibes}
%\usetheme{Bergen}
%\usetheme{Berkeley}
%\usetheme{Berlin}
%\usetheme{Boadilla}
%\usetheme{CambridgeUS}
%\usetheme{Copenhagen}
%\usetheme{Darmstadt}
%\usetheme{Dresden}
%\usetheme{Frankfurt}
%\usetheme{Goettingen}
%\usetheme{Hannover}
%\usetheme{Ilmenau}
%\usetheme{JuanLesPins}
%\usetheme{Luebeck}
\usetheme{Madrid}
%\usetheme{Malmoe}
%\usetheme{Marburg}
%\usetheme{Montpellier}
%\usetheme{PaloAlto}
%\usetheme{Pittsburgh}
%\usetheme{Rochester}
%\usetheme{Singapore}
%\usetheme{Szeged}
%\usetheme{Warsaw}

% As well as themes, the Beamer class has a number of color themes
% for any slide theme. Uncomment each of these in turn to see how it
% changes the colors of your current slide theme.

%\usecolortheme{albatross}
%\usecolortheme{beaver}
%\usecolortheme{beetle}
%\usecolortheme{crane}
%\usecolortheme{dolphin}
%\usecolortheme{dove}
%\usecolortheme{fly}
%\usecolortheme{lily}
%\usecolortheme{orchid}
%\usecolortheme{rose}
%\usecolortheme{seagull}
%\usecolortheme{seahorse}
%\usecolortheme{whale}
%\usecolortheme{wolverine}

%\setbeamertemplate{footline} % To remove the footer line in all slides uncomment this line
%\setbeamertemplate{footline}[page number] % To replace the footer line in all slides with a simple slide count uncomment this line

%\setbeamertemplate{navigation symbols}{} % To remove the navigation symbols from the bottom of all slides uncomment this line
}

\usepackage{graphicx} % Allows including images
\usepackage{booktabs} % Allows the use of \toprule, \midrule and \bottomrule in tables
\usepackage{algorithm2e}
\usepackage{algorithmic}
%----------------------------------------------------------------------------------------
%	TITLE PAGE
%----------------------------------------------------------------------------------------


\title[Euclidean Algorithm]{Exploring the Euclidean Algorithm} % The short title appears at the bottom of every slide, the full title is only on the title page

\author{Stephen Capps, Sarah Sahibzada, \& Taylor Wilson} % Your name
\institute[TAMU] % Your institution as it will appear on the bottom of every slide, may be shorthand to save space
{
Texas A\&{}M University \\ % Your institution for the title page
\medskip
\textit{Supervisor: Sara Pollock} % Your email address
}
\date{\today} % Date, can be changed to a custom date

\begin{document}

\begin{frame}
\titlepage % Print the title page as the first slide
\end{frame}

\begin{frame}
\frametitle{Overview} % Table of contents slide, comment this block out to remove it
\tableofcontents % Throughout your presentation, if you choose to use \section{} and \subsection{} commands, these will automatically be printed on this slide as an overview of your presentation
\end{frame}

%----------------------------------------------------------------------------------------
%	PRESENTATION SLIDES
%----------------------------------------------------------------------------------------

%------------------------------------------------


\section{Euclidean Algorithm Iterations and Results}

\subsection{What to explore}
\begin{frame}
Next, we decided to explore the distributions of these iterations:
Do most pairs take many iterations? What is the distribution?

The following graphs are the answer to these questions.


\end{frame}
%----------------------------------------------------------------------------------------
\section{Introduction to Complexity Theory}
\begin{frame}
\frametitle{Introduction to Complexity Theory}
\indent Insofar as this course focuses on algorithms, it is first necessary to define the term 'algorithm', as well as several other auxiliary terms used across this presentation.
It is necessary to define some sort of \underline{machine-independent} metric for algorithm running time, so it can be universally used. It is also necessary to define some auxiliary terms to do this:
\begin{itemize}
\item An \underline{algorithm} is a well-defined computational procedure that takes a variable input and halts with an output. (Oorschot and Vanstone, A Handbook of Applied Cryptography)
\item The \underline{running-time} of an algorithm is a function of the number of single-bit operations it takes for an algorithm to complete. 
\item The \underline{size} of an input is the number of bits in the input, or the number of inputs, to an algorithm.
\item The \underline{execution time} of an algorithm's implementation is taken here to mean the average amount of time in nanoseconds for a particular implementation of an algorithm to run. 
\end{itemize}
\end{frame}
\begin{frame}
\frametitle{A Crash Course in Algorithmic Analysis}
\indent Often, the run-time functions can grow to be complex and involve multiple factors. It is often enough to know that a particular algorithm "grows proportionally to" some value (Goodrich, Data Structures and Algorithms in C++). This is known as \underline{asymptotic notation}. \\
\begin{itemize}
\item Let f(n) and g(n) be functions such that $f(n): \mathbb{Z}->\mathbb{R}$ and $g(n): \mathbb{Z}->\mathbb{R}$. Then we say that f(n) is O(g(n)) iff $\exists$ c \textgreater $0$ and n $\in$ $\mathbb{Z}$ such that 
f(n) $\leq$ cg(n), for $\forall$ n' $\geq$ n 
\item This is known as Big-O notation. It allows us to ignore constant factors and lower-order terms, focusing only on the highest-order term--the one which governs its asymptotic behavior.
\item In general, there are six functions fundamental to algorithmic analysis: log(n) \textless n \textless n log(n) \textless $n^2$ \textless $n^3$ \textless $2^n$ (Goodrich, Data Structures and Algorithms in C++)
\item Hence, this is known as asymptotic notation.
\end{itemize}
\end{frame}
\begin{frame}
\frametitle{Bit-Complexity of Integer Operations}
\indent In the ring of integers, we define the bit complexity of addition and multiplication, as well as their inverses, as below: \\ 
\begin{tabular}{|r|l|}
\hline
Operation & Bit-Complexity \\  \hline
+ & O(log(a) + log(b)) = O(log(n)) \\  \hline
- & O(log(a) + log(b)) = O(log(n)) \\ \hline
* & O(log(a) * log(b)) = O($log(n)^2$) \\ \hline
\textbackslash & O(log(a) * log(b)) = O($log(n)^2$) \\ \hline

\end{tabular} \\
\indent These are fundamental values; they allow a baseline computation for running-time complexity. Things like recursion, which incur an extra execution time penalty for function calls, are one of many factors that can confound execution time. \\
\end{frame}
\section{Euclidean Algorithm Variations}
\begin{frame}
\frametitle{Euclidean Algorithm Variations}
\indent Three iterative versions, two recursive versions, and two binary versions of this algorithm were implemented. 
\begin{itemize}
\item Binary GCD algorithm will run with O($log(b)^2$) bit complexity, where b represents the maximum number of bits in the integer inputs.
\item The iterative modular algorithm will run with O(log(u) * log(v)), where u and v are integer inputs. The execution time may be longer, due to an added penalty for function calls.
\item The least-remainders Euclidean algorithm will run in, at the worst case, O(log(u) * log(v)). (Bach and Shallit, Algorithmic Number Theory).
\end{itemize}
\end{frame}
\section{The Algorithms}

\begin{frame}
\frametitle{Iterative, Subtractive}
\begin{algorithm}[H]
\begin{algorithmic}
\STATE{Input: integers U, V}
\STATE{Output: GCD(U,V)}
\IF{ U \textgreater V}
	\STATE{swap(U,V)}
\ENDIF
\WHILE{U $\neq$ V}
	\IF{U \textgreater V}
	\STATE{U <- U - V}
	\ELSE
		\STATE{V = V - U}
	\ENDIF
\ENDWHILE

\STATE{return U}
\end{algorithmic}
\end{algorithm}
\end{frame}

\begin{frame}
\frametitle{Iterative, Modular}
\begin{algorithm}[H]
\begin{algorithmic}
\STATE{Input: integers U, V}
\STATE{Output: GCD(U,V)}
\IF{ V \textgreater U}
	\STATE{swap(U,V)}
\ENDIF
\WHILE{U $\neq$ V}
	\STATE{(U,V) \textless - (V, U MOD V)}
\ENDWHILE
\STATE{return U}
\end{algorithmic}
\end{algorithm}
\indent (Bach and Shallit, Algorithmic Number Theory)
\end{frame}



\begin{frame}
\frametitle{Recursive, Least Remainder}
\begin{algorithm}[H]
\begin{algorithmic}
\STATE{Input: integers U, V}
\STATE{Output: GCD(U,V)}
\IF{ V == 0 }
\STATE{return V}
\ELSE
\STATE{return LeastRemainderEuclidean(V, U MOD V)}
\ENDIF

\end{algorithmic}
\end{algorithm}
\end{frame}

\begin{frame}
\frametitle{Auxiliary Functions for Least Remainder}
\begin{algorithm}[H]
\begin{algorithmic}
\STATE{Input: integers U,V}
\STATE{Output: "Round Mod" function}
\IF{V == 0}
\STATE{return U}
\ELSE
\STATE{Return U = (V * ROUND(U / V))}
\ENDIF
\end{algorithmic}
\end{algorithm}

\begin{algorithm}[H]
\begin{algorithmic}
\STATE{Input: integers U to round}
\STATE{Output: Integer round function}
\STATE{Frac = U - $0.5$}
\STATE{Return CEIL(Frac) as Integer}

\end{algorithmic}
\end{algorithm}
\end{frame}
\begin{frame}
\frametitle{Binary GCD}
\begin{algorithm}[H]
\begin{algorithmic}
\STATE g \textless - $1$
\WHILE{U, V are even}
	\STATE{U \textless - $U/2$}
	\STATE{V \textless - $V/2$}
	\STATE{ g = 2g } 
\ENDWHILE

\WHILE{U $\neq$ 0}
	\IF{U is even}
		\STATE{U \textless - $U/2$}
	\ELSIF{V is even}
		\STATE{V \textless - $V/2$}
	\ELSE
		\STATE{t \textless - abs(U-V)/2}
		\IF{U \textless V}
			\STATE{V \textless - t}
		\ENDIF		
	\ENDIF
\ENDWHILE
\end{algorithmic}
\end{algorithm}
\end{frame}

\begin{frame}
\frametitle{Binary GCD cont'd}
\begin{algorithm}[H]
\begin{algorithmic}
		\STATE{else}
			\STATE{U \textless - T}
			\STATE{MAX(U,V) = abs(U,V)/2}
		
\STATE{return g*V}

\end{algorithmic}
\end{algorithm}
\end{frame}


\section{Implementation Detail: Java BigInteger Library}
\begin{frame}
\frametitle{Implementation Detail: Java BigInteger Library}
\begin{itemize} 
\item In order to test these algorithms against large numbers, the Java BigInteger Library was used due to its support for arbitrary-precision arithmetic and numbers.
	\begin{itemize}
	\item A BigInteger is an abstraction for an arbitrary-length bit string
	\item In-built GCD function against which our algorithms were compared
	\item No in-built random number generator; used Java's random number 	generator to generate individual bits in a string and wrap these into a 	BigInteger
	\end{itemize}
\item Hierarchical implementation: parent Euclidean Algorithm class with Iterative Algorithm and Recursive Algorithm derived classes; each implementation was a subclass of these
\item In all cases, the corresponding BigInteger arithmetic operation was used; for the binary GCD algorithm, both the arithmetic operations and functions which performed direct bit shifting, were used.
\item All data were collected on a Linux-based server hosted through the Texas A\&M University Department of Computer Science. This process was automated with scripts in Bash.
\end{itemize}
\end{frame}
\section{Baseline Run-Times for BigInteger Library Functions}
\begin{frame}
\frametitle{Baseline Run-Times for BigInteger Library Functions}
\indent In order to effectively measure the running-time of each algorithm, some data on the execution-time of the relevant BigInteger functions were collected, as well as the execution time of our random number generation function: \\
\begin{tabular}{|l|r|}
\hline
Function & Average Time (ns) \\ \hline
Random Number Generation & 1.9e5 ns \\ \hline
Addition & 1.3e5 ns \\ \hline
Subtraction & 1.0e5 ns \\ \hline
Multiplication & 1.4e5 ns \\ \hline
Modulus & 1.4e5 ns \\ \hline
Division & 1.7e5 ns \\ \hline
Left Shift & 3.9e2 ns \\ \hline
Right Shift & 4.7e2 ns \\ \hline
Absolute Value & 2.3e2 ns \\ \hline
Assignment & 2.1e2 ns \\ \hline

\end{tabular}
\\ \indent All data measured was adjusted to account for the costs of absolute value, random number generation, et cetera.
\end{frame}
\section{Complexity Analysis Results}
\begin{frame}
\frametitle{Complexity Analysis Results}
The resultant execution times for each algorithm are given below:

\begin{tabular}{|l|l|}
\hline
	\textbf{Algorithm} & \textbf{Execution Time}
	\\
	\hline Extended & 6.09e9 ns\\
	\hline Iterative, Mod & 4.35e9 ns\\
	\hline Iterative, Subtractive & 1.91e9 ns\\
	\hline Recursive, Mod & 4.77e9 ns\\
	\hline Recursive, Least Positive Remaidner
 & 11.8e9 ns\\
	\hline Stein’s Binary Algorithm, BigInteger Operations & 6.04e9 ns\\
	\hline Stein’s Binary Algorithm, Bit Shifting Operations & 4.68e9 ns\\
	\hline Java BigInteger GCD & 1.30e9 ns\\
	\hline
\end{tabular}

\end{frame}

\begin{frame}
\frametitle{Discussion}
\begin{itemize}
\item A large number of runs for each algorithm were averaged; time elapsed was calculated using the Java System.nanoTime() timer, and adjusted for superfluous costs 
\item Subtractive was among the fastest; lower bitwise cost?
\item Direct bit shifting was also among the fastest; working directly with hardware
\item Number of iterations varied: direct bit shifting was the most expensive bitwise
\item No strong correlation between number of iterations and execution time: least positive remainder is the quickest in terms of iterations, but one of the slower in execution time
\end{itemize}
\end{frame}

\section{Future Work In This Area}
\begin{frame}
\frametitle{Future Work In This Area}
\begin{itemize}
\item All Java programs are run on a virtual machine, allowing a great degree of platform independence; future repetitions will not see a significant change in execution time.
\item Future computational work might consider consolidations of less-expensive bitwise operations rather than consolidations of steps (in the vein of the modulus function)
\item Consider parallel implementations of GCD algorithms
\end{itemize}
\end{frame}
\section{Introduction to Neural Networks}

\section{Attempts at using Neural Networks}

\section{Neural Networks Results}
\end{document} 
